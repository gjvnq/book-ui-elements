%!TEX TS-program = xelatex
%!TEX program = xelatex
%!TEX output_directory = build
%!TEX spellcheck=pt_BR

\documentclass[
	% -- opções da classe memoir --
	12pt,				% tamanho da fonte
	openright,			% capítulos começam em pág ímpar (insere página vazia caso preciso)
	twoside,			% para impressão em recto e verso. Oposto a oneside
	a4paper,			% tamanho do papel
	english,			% idioma adicional para hifenização
	french,				% idioma adicional para hifenização
	brazil,				% o último idioma é o principal do documento
	sumario=tradicional
]{abntex2}

% compilação de fontes
\usepackage{amsmath}
\usepackage{mathtools}
\usepackage{amsfonts}
\usepackage{mathrsfs} % para mathscr
\usepackage[subentrycounter,seeautonumberlist,nonumberlist=true,toc]{glossaries}

  % % se for utilizar as fontes do sistema: **escolha sua fonte**
    % comandos de fontes
\usepackage{mathspec}
\usepackage{fontspec}
\setmathsfont(Digits,Latin,Greek){Source Serif Pro}
\setmathrm{Source Serif Pro}
\setmainfont[Numbers=OldStyle]{Source Serif Pro} %fonte principal (serifada)
\setsansfont[Scale=0.9]{Source Sans Pro} %fonte sem serifas
\setmonofont[Scale=MatchLowercase]{Source Code Pro} % fonte monoespaçada

\usepackage{polyglossia} %always load polyblossia after fonts for digits in math mode
\setmainlanguage{brazil}
\setotherlanguages{french,english,spanish,german,italian}

\newfontfamily\ipafont{Doulos SIL}
\newcommand{\ipa}[1]{{\ipafont #1}}

\usepackage{marginnote}

%% Observação: o pacote polyglossia pode apresentar erro ao ser utilizado com ifxetex + babel.
%% Se isso acontecer, atualize o pacote para a versão mais recente ou utilize somente uma das sequências (pdflatex ou xelatex), comentando ou apagando a outra.

\usepackage{microtype} 				% para melhorias de justificação
\usepackage[dvipsnames]{xcolor} 		% para cores
\usepackage{graphicx} 			% para imagens
\usepackage{booktabs,tabularx,rotating}	% para tabelas
\usepackage{mdframed} 				% para caixas de texto como na CIP do verso do título
\usepackage{multicol}				% tabelas com colunas mescladas
\usepackage{lettrine}				% letras capitulares
\usepackage{xspace} 				% para nao precisar de espaços com {} depois de comandos
									% como \LaTeX e abreviações criadas pelo usuário
\usepackage{lipsum} 				% para texto de preenchimento de exemplo
\usepackage{leading}				% espaçamento entrelinhas (leading)
\leading{13pt}

\usepackage{tikz}
\usetikzlibrary{matrix, backgrounds,fit}

\usepackage{siunitx}
% ---
% Pacotes de citações
% ---
\usepackage[backend=biber,style=alphabetic,citestyle=alphabetic,backref=true]{biblatex}
\bibliography{refs}
\DefineBibliographyStrings{brazil}{
  backrefpage = {citado na página},
  backrefpages = {citado nas páginas},
}

% ---
% Informações do documento
% ---
\titulo{Computadores e suas Abstrações}
\autor{Gabriel V. N. de Queiroz}
\data{2019, v<VERSION>}
\preambulo{Breve sinopse do livro}
\local{São Paulo}
\instituicao{Publicações Acadêmicas Ltda.\\ \abnTeX\ v<VERSION>}

% alterando o aspecto da cor azul
\definecolor{blue}{RGB}{41,5,195}

% informações do PDF
\makeatletter
\hypersetup{
     	%pagebackref=true,
		pdftitle={\@title},
		pdfauthor={\@author},
    	pdfsubject={\imprimirpreambulo},
	    pdfcreator={LaTeX with abnTeX2},
		pdfkeywords={abnt}{latex}{abntex}{abntex2}{livro},
		colorlinks=false,       		% false: boxed links; true: colored links
    	linkcolor=blue,          	% color of internal links
    	citecolor=blue,        		% color of links to bibliography
    	filecolor=magenta,      		% color of file links
		urlcolor=blue,
		bookmarksdepth=4
}
\makeatother
\usepackage{xurl}
% ---


% ---
% Estilo de capítulos
%
% \chapterstyle{pedersen}
% \chapterstyle{lyhne}
%\chapterstyle{madsen}
\chapterstyle{veelo}
%
% Veja outros estilos em:
% https://www.ctan.org/tex-archive/info/MemoirChapStyles
% ---

% para cabeçalhos sem estar em maiúsculas
%\nouppercaseheads

% -----
% Declarações de cabecalhos
% -----
% Cabecalho padrao
\makepagestyle{abntbookheadings}
\makeevenhead{abntbookheadings}{\ABNTEXfontereduzida\thepage}{}{\ABNTEXfontereduzida\textit\leftmark}
\makeoddhead{abntbookheadings}{\ABNTEXfontereduzida\textit\rightmark}{}{\ABNTEXfontereduzida\thepage}
\makeheadrule{abntbookheadings}{\textwidth}{\normalrulethickness}

% Cabecalho do inicio do capitulo
\makepagestyle{abntbookchapfirst}
\makeoddhead{abntbookchapfirst}{}{}{}

% Configura layout para elementos textuais
\renewcommand{\textual}{%
  \pagestyle{abntbookheadings}%
  \aliaspagestyle{chapter}{abntbookchapfirst}% customizing chapter pagestyle
  \nouppercaseheads%
  \bookmarksetup{startatroot}%
}
% ---



% Margens do documento
%% (margens do abntex2 não combinam nem com A5 nem com estilos de capítulo da
% classe memoir.)
\setlrmarginsandblock{2.5cm}{3.5cm}{*}
\setulmarginsandblock{2.5cm}{3.5cm}{*}
\checkandfixthelayout
% ---

% Termos indexados
\usepackage{makeidx}
\makeindex
% \providecommand{\myglsindextext}[1]{\glsentrydesc{#1} (\glsentrytext{#1})}
% \providecommand{\myglsindex}[1]{\index{\myglsindextext{#1}}}
\renewcommand{\glsnamefont}[1]{{\normalfont \textsc{#1}}}
\newcommand{\glsstyle}[1]{{\textsc{#1}}}
\providecommand{\myglsindex}[1]{\index{#1@\glsstyle{#1}}}
\providecommand{\mygls}[1]{{\scshape \gls{#1}\myglsindex{#1}}}
\providecommand{\myglspl}[1]{{\scshape \glspl{#1}\myglsindex{#1}}}
\providecommand{\myGls}[1]{{\scshape \Gls{#1}\myglsindex{#1}}}
\providecommand{\myGlspl}[1]{{\scshape \Glspl{#1}\myglsindex{#1}}}
\providecommand{\myglsIdx}[2]{{\scshape \gls{#1}\myglsindex{#2}}}
\providecommand{\myglsplIdx}[2]{{\scshape \glspl{#1}\myglsindex{#2}}}
\providecommand{\myGlsIdx}[2]{{\scshape \Gls{#1}\myglsindex{#2}}}
\providecommand{\myGlsplIdx}[2]{{\scshape \Glspl{#1}\myglsindex{#2}}}


% Números
\newcommand{\zb}{0\texttt{b}} % 0b prefix
\newcommand{\zx}{0\texttt{x}} % 0x prefix
\newcommand{\nx}[1]{0\texttt{x}\mathrm{#1}} % 0x prefix
\newcommand{\numfmt}[1]{\texttt{#1}}
\newcommand\ldiv[2]{% long division
$\strut#1$\kern.25em\smash{\raise.3ex\hbox{$\big)$}}$\mkern-8mu
        \overline{\enspace\strut#2}$}

% Glossários
\makeglossaries
% \newglossaryentry{pai}{
%     name={pai},
%     plural={pai},
%     description={este é uma entrada pai, que possui outras
%     subentradas.} }
\newglossaryentry{bit}{
    name={bit},
    plural={bits},
    description={Dígito binário. Pode ser \texttt{0} (zero) ou \texttt{1} (um)} }
\newglossaryentry{palavra}{
    name={palavra},
    plural={palavras},
    description={(do inglês: \mygls{word}) Sequência de bits cujo tamanho é fixo e depende do processador em questão} }
\newglossaryentry{word}{
    name={word},
    plural={words},
    description={Veja \mygls{palavra}} }

\newglossaryentry{file}{
    name={file},
    plural={files},
    description={Veja \mygls{ficheiro}} }
\newglossaryentry{archive}{
    name={archive},
    plural={archives},
    description={Veja \mygls{arquivário}} }
\newglossaryentry{arquivo}{
    name={arquivo},
    plural={arquivos},
    description={Veja \mygls{ficheiro} e \mygls{arquivário}} }
\newglossaryentry{ficheiro}{
    name={ficheiro},
    plural={ficheiros},
    description={Unidade discreta de armazenamento geralmente organizada em pastas de forma hierárquica} }
\newglossaryentry{arquivário}{
    name={arquivário},
    plural={arquivários},
    description={(neologismo, do inglês: \mygls{archive}, mais frequentemente traduzido como \mygls{arquivo}) Coleção de registros, informações ou documentos. Ex: \emph{Arquivo Nacional}, \emph{Internet Archive} e \emph{National Archives and Records Administration}} }
\newglossaryentry{pasta}{
    name={pasta},
    plural={pastas},
    description={(do inglês: \mygls{folder}) Coleção de arquivos e outras pastas} }
\newglossaryentry{folder}{
    name={folder},
    plural={folders},
    description={Veja \mygls{pasta}} }
\newglossaryentry{diretório}{
    name={diretório},
    plural={diretórios},
    description={Veja \mygls{pasta}} }
\newglossaryentry{directory}{
    name={directory},
    plural={directorys},
    description={Veja \mygls{folder}} }

\setlrmarginsandblock{3cm}{6cm}{*}
\setulmarginsandblock{3cm}{6cm}{*}
\checkandfixthelayout
\setlength{\marginparwidth}{\marginparwidth-1cm}
\renewcommand*{\marginpar}[1]{
            \checkoddpage
            \ifoddpage
               \marginparmargin{right}
            \else
               \marginparmargin{right}%left
            \fi
            \marginnote{#1}}

% ---
% Início do documento
% ---
\begin{document}
\addcontentsline{toc}{chapter}{Capa}
\frenchspacing

\frontmatter

% ---
% Contra-capa
% ---
\begin{titlingpage}

\phantom{xxx}
\vspace{0.5cm}
\huge
\raggedright
\imprimirautor\\
\vspace{2.5cm}
\huge
{\raggedleft
% \includegraphics[scale=0.9]{abntex2-modelo-img-marca.pdf}\\[1cm]
\textit{\textcolor{blue}{\imprimirtitulo}}\\[1cm]
}
\centering
\vfill
\Large
\imprimirinstituicao

% Verso da contra-capa
\clearpage
\ABNTEXfontereduzida
© 2017 \imprimirautor \space \& \imprimirinstituicao

%Qualquer parte desta publicação pode ser reproduzida, desde que citada a fonte.

\vspace*{\fill}

\begin{center}
Dados Internacionais de Catalogação na Publicação (\textsc{cip})
Câmara Brasileira do Livro, \textsc{sp}, Brasil
\end{center}

\begin{mdframed}
\noindent Queiroz, Gabriel V. N. de

\imprimirtitulo. / \imprimirautor. -- \imprimirlocal: \imprimirinstituicao
Ltda., 2015.

\medskip

Bibliografia.

ISBN XXXX-XXXX-XX.

\medskip

1. Programas de computador. 2. Tipografia. 3. Latex. 4. Normas ABNT.

\end{mdframed}

\end{titlingpage}

% ---
% inserir lista de ilustrações
% ---
\pdfbookmark[0]{\listfigurename}{lof}
\listoffigures*
\cleardoublepage

% ---
% inserir lista de tabelas
% ---
\pdfbookmark[0]{\listtablename}{lot}
\listoftables*
\cleardoublepage
% ---

% ---
% inserir o sumario
% ---
\pdfbookmark[0]{\contentsname}{toc}
\tableofcontents*
\cleardoublepage
% ---

% ------------------------------------------------------------
% Início da parte textual
% ------------------------------------------------------------
%\textual
\mainmatter
% ------------------------------------------------------------

% ------------------------------------------------------------
\chapter*[Introdução]{Introdução}
\addcontentsline{toc}{chapter}{Introdução}
% ------------------------------------------------------------

\lettrine[nindent=0.35em,lhang=0.40,loversize=0.3]{E}{ste documento} faz parte
do projeto \abnTeX\footnote{\url{http://www.abntex.net.br/}}, e destina-se
a servir de modelo para composição e diagramação de livros e folhetos em
\LaTeX em conformidade com a norma ABNT NBR 6029:2006 \emph{Informação e
documentação - Livros e folhetos - Apresentação}.



% ------------------------------------------------------------
\chapter{Informação}
% ------------------------------------------------------------

\section{Números Inteiros}

\lettrine[nindent=0.35em,lhang=0.40,loversize=0.3]{N}{úmeros}
são comumente representados em um sistema de base dez com os seguintes dez dígitos indo-arábicos: 0, 1, 2, 3, 4, 5, 6, 7, 8 e 9.

Por diversos motivos, computadores comumente representam números em um sistema binário com apenas dois dígitos: 0 e 1. Cada dígito binário é comumente chamado de \mygls{bit}.

O número cento e vinte e três é comumente escrito como $123$ em base dez. A expressão $123$ equivale a dizer: um vezes cem mais dois vezes dez mais três vezes um. Note a multiplicação implícita por potências de base 10.

Em binário, escrevemos cento e vinte e três como $1111011$. Para evitar ambiguidades quanto a base, podemos usar a notação de parênteses $(1111011)_{\color{red}2}$ com a base em base dez no subíndice, aqui destacado em vermelho. Ou podemos utilizar a notação de prefixo $\zb1111011$, onde a letra $\texttt{b}$ indica que se trata de um número em \underline{b}inário.

Em símbolos:

\[
	123 = 1 \times 10^2 + 2 \times 10^1 + 3 \times 10^0
\]
\[
	\zb1111011 = 1 \times 2^6 + 1 \times 2^5 + 1 \times 2^4 + 1 \times 2^3 + 0 \times 2^2 + 1  \times 2^1 + 1  \times 2^0
\]

\marginpar{Você não precisa ser bom em calcular conversões de bases, você só precisa entender a ideia geral de diferentes bases de numeração.}

O processo de conversão de decimal para binário consiste em divisões por dois e usar os restos ao contrário. Exemplo (os restos estão em círculos para facilitar a leitura):

\begin{tikzpicture}
\matrix (a) [matrix of math nodes,column sep=2pt]
{
123 &  & \phantom{0}2 & & & & & & & & \\
\phantom{00}\textcircled{1} &  & \phantom{0}61 &  \phantom{00}2 & & & & & & & \\
    &  &  \phantom{00}\textcircled{1} & \phantom{0}30 &  \phantom{00}2 & & & & & & \\
    &  &    &  \phantom{00}\textcircled{0} & \phantom{0}15 & \phantom{00}2 & & & & & \\
    &  &    &    &  \phantom{00}\textcircled{1} & \phantom{00}7 & \phantom{00}2 & & & & \\
    &  &    &    &    & \phantom{00}\textcircled{1} & \phantom{00}3 & \phantom{00}2 & & & \\
    &  &    &    &    &   & \phantom{00}\textcircled{1} & \phantom{00}1 & \phantom{00}2 & & \\
    &  &    &    &    &   &   & \phantom{00}\textcircled{1} & \phantom{00}0 & & \\
};
\draw (a-1-3.north west)|-(a-1-3.south east);
\draw (a-2-4.north west)|-(a-2-4.south east);
\draw (a-3-5.north west)|-(a-3-5.south east);
\draw (a-4-6.north west)|-(a-4-6.south east);
\draw (a-5-7.north west)|-(a-5-7.south east);
\draw (a-6-8.north west)|-(a-6-8.south east);
\draw (a-7-9.north west)|-(a-7-9.south east);
\end{tikzpicture}

\[
	123 = \zb1111011
\]

Grupos de \myglspl{bit} de tamanho regular recebem o nome de \mygls{palavra}. Por convenção, \myglspl{palavra} de 8 bits recebem o nome de \mygls{byte} (en: /\ipa{baɪt}/, pt-BR: /\ipa{baitʃi}/).

Além do sistema binário, é comum utilizarmos o sistema hexadecimal (base 16) e, mais raramente, o sistema octal (base 8) para reduzir o número de dígitos necessários (leia-se: para escrevermos menos).

Como o sistema hexadecimal precisa de mais dígitos do que há no sistema indo arábico, utilizamos as primeiras letras do alfabeto para os dígitos maiores que nove. Logo, os dígitos hexadecimais são: 0, 1, 2, 3, 4, 5, 6, 7, 8, 9, A, B, C, D, E, F.

Os dígitos A, B, D, E, F correspondem aos números dez, onze, doze, treze, catorze e quinze respectivamente.

Similarmente aos números binários, podemos usar as seguintes notações para evitar ambiguidade: $(\mathrm{F})_{16}$ e $\nx{F}$.

A forma mais simples de convertermos binário para hexadecimal é agrupando os \myglspl{bit} de quatro em quatro.

\[
	123 = \zb1111011 = \zb\underbrace{0111}_{7}\;\underbrace{1011}_{11} = \nx{7B}
\]

Outro exemplo:

\begin{align*}
	&\nx{ABC} = \nx{A}\times16^2 + \nx{B}\times16^1 + \nx{C}\times16^0 \\
	&= 10\times16^2 + 11\times16^1 + 12\times16^0 = 2748
\end{align*}

\section{Números de Ponto Flutuante}

\lettrine[nindent=0.35em,lhang=0.40,loversize=0.3]{Q}{uando}
queremos representar números que cobrem várias ordens de grandeza, é comum usarmos notação científica para aproximá-los.

Em computação, chamamos essa técnica de \myglspl{float}. Esse termo é frequentemente abreviado para \myglsplIdx{num-flu}{float} ou \myglsplIdx{lit-float}{float}.

\marginpar{Visto que, no Brasil, usamos vírgulas como separador decimal, talvez o nome número de \emph{vírgula} flutuante fosse mais adequado.}


% ------------------------------------------------------------
\chapter{Exemplos de imagens}
% ------------------------------------------------------------

\lipsum[1]

\begin{figure}
\centering
\includegraphics[width=0.6\linewidth]{example-image-a}
\caption{Exemplo de imagem.}
\label{fig:exemplo}
\end{figure}

\lipsum[6]





\lipsum[7]

% ------------------------------------------------------------
\chapter{Exemplos de tabela}
% ------------------------------------------------------------

\section{Uma seção}

\lipsum[8]

\begin{table}
\caption{Pequeno vocabulário de design de livros\label{vocabulario-texto}}
\ABNTEXfontereduzida
\begin{tabular}{p{4cm}p{4cm}}
\toprule
\textit{Termo em inglês} & \textit{Termo em português}\\
\midrule
\ABNTEXfontereduzida
title page & folha de rosto.\\

cover & capa\\

back cover & quarta capa ou contra-capa ou verso da capa\\

bastard title ou half title & falsa folha de rosto. Tem só o título do livro.\\

table of contents & sumário\\

text block ou book block & miolo\\

print space (alemão: \textit{Satzspiegel}) & mancha gráfica\\

section, gathering, quire (especialmente se não impresso), signature & caderno\\

leaf = folio (latim) & folha, composta de recto (lat.) (anverso/frente) e verso (lat.) (verso). Geralmente o recto é página ímpar, e verso é página par.\\

hardcover & capa dura.\\

endpaper/endsheet & folha de guarda. Folha de papel para prender o miolo do livro na capa dura.\\

dust jacket, dust cover, book jacket, dust wrapper & sobrecapa. Geralmente de papel, para cobrir capas duras.\\

front matter & parte pré-textual.\\

main matter & parte textual\\

back matter & parte pós-textual. Composta por epílogo, posfácio, apêndice, glossário, bibliografia, índice remissivo (inglês: index), colofão etc.\\

colophon & colofão. Breve descrição sobre aspectos da publicação do livro. \\

running headers & títulos correntes\\

volume & volume. Conjunto de páginas encadernadas.\\

\bottomrule
\end{tabular}
\footnotesize Fontes:\\
\url{http://pt.wikipedia.org/wiki/Design_de_livros}\\
\url{http://en.wikipedia.org/wiki/Book_design}\\
\url{http://static.lexicool.com/dictionary/RX7KW614433.pdf}\\
\end{table}


\begin{table}
\caption{Exemplo de tabela utilizando o pacote \textsf{booktabs}.}
\centering
\begin{tabular}{llr}
\toprule
\multicolumn{2}{c}{Item} \\
\cmidrule(r){1-2}
Animal    & Description & Price (\$) \\
\midrule
Gnat      & per gram    & 13.65      \\
          & each        & 0.01       \\
Gnu       & stuffed     & 92.50      \\
Emu       & stuffed     & 33.33      \\
Armadillo & frozen      & 8.99       \\
\bottomrule
\multicolumn{3}{l}{\ABNTEXfontereduzida Fonte: \url{http://en.wikibooks.org/wiki/LaTeX/Tables}}
\end{tabular}
\end{table}

\lipsum[9]
\cite{posix}
\cite{dirVsFolder}

% ------------------------------------------------------------
\postextual % pós-textual
% ------------------------------------------------------------

\phantompart
% \providetranslation{Glossary}{Glossário}
% \providetranslation{Acronyms}{Siglas}
% \providetranslation{Notation (glossaries)}{Notação}
% \providetranslation{Description (glossaries)}{Descrição}
% \providetranslation{Symbol (glossaries)}{Símbolo}
% \providetranslation{Page List (glossaries)}{Lista de Páginas}
% \providetranslation{Symbols (glossaries)}{Símbolos}
% \providetranslation{Numbers (glossaries)}{Números}
% \setglossarystyle{tree}

% \cleardoublepage
\phantomsection
% \addcontentsline{toc}{chapter}{\glossaryname}
\glsaddall
\printglossary[title=Glossário]

\printindex

% ------------------------------------------------------------
\printbibliography
% ------------------------------------------------------------

\end{document}