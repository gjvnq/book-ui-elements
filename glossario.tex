% Glossários
\makeglossaries
% \newglossaryentry{pai}{
%     name={pai},
%     plural={pai},
%     description={este é uma entrada pai, que possui outras
%     subentradas.} }
\newglossaryentry{bit}{
    name={bit},
    plural={bits},
    description={Dígito binário. Pode ser \texttt{0} (zero) ou \texttt{1} (um)} }
\newglossaryentry{palavra}{
    name={palavra},
    plural={palavras},
    description={(do inglês: \mygls{word}) Sequência de bits cujo tamanho é fixo e depende do processador em questão} }
\newglossaryentry{word}{
    name={word},
    plural={words},
    description={Veja \mygls{palavra}} }

\newglossaryentry{file}{
    name={file},
    plural={files},
    description={Veja \mygls{ficheiro}} }
\newglossaryentry{archive}{
    name={archive},
    plural={archives},
    description={Veja \mygls{arquivário}} }
\newglossaryentry{arquivo}{
    name={arquivo},
    plural={arquivos},
    description={Veja \mygls{ficheiro} e \mygls{arquivário}} }
\newglossaryentry{ficheiro}{
    name={ficheiro},
    plural={ficheiros},
    description={Unidade discreta de armazenamento geralmente organizada em pastas de forma hierárquica} }
\newglossaryentry{arquivário}{
    name={arquivário},
    plural={arquivários},
    description={(neologismo, do inglês: \mygls{archive}, mais frequentemente traduzido como \mygls{arquivo}) Coleção de registros, informações ou documentos. Ex: \emph{Arquivo Nacional}, \emph{Internet Archive} e \emph{National Archives and Records Administration}} }
\newglossaryentry{pasta}{
    name={pasta},
    plural={pastas},
    description={(do inglês: \mygls{folder}) Coleção de arquivos e outras pastas} }
\newglossaryentry{folder}{
    name={folder},
    plural={folders},
    description={Veja \mygls{pasta}} }
\newglossaryentry{diretório}{
    name={diretório},
    plural={diretórios},
    description={Veja \mygls{pasta}} }
\newglossaryentry{directory}{
    name={directory},
    plural={directorys},
    description={Veja \mygls{folder}} }