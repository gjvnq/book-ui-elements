% compilação de fontes

\usepackage{amsmath}
\usepackage{mathtools}
\usepackage{amsfonts}
\usepackage{mathrsfs} % para mathscr
\usepackage[subentrycounter,seeautonumberlist,nonumberlist=true,toc]{glossaries}

  % % se for utilizar as fontes do sistema: **escolha sua fonte**
    % comandos de fontes
\usepackage{mathspec}
\usepackage{fontspec}
\setmathsfont(Digits,Latin,Greek){Source Serif Pro}
\setmathrm{Source Serif Pro}
\setmainfont[Numbers=OldStyle]{Source Serif Pro} %fonte principal (serifada)
\setsansfont[Scale=0.9]{Source Sans Pro} %fonte sem serifas
\setmonofont[Scale=MatchLowercase]{Source Code Pro} % fonte monoespaçada

\usepackage{polyglossia} %always load polyblossia after fonts for digits in math mode
\setmainlanguage{brazil}
\setotherlanguages{french,english,spanish,german,italian}

\newfontfamily\ipafont{Doulos SIL}
\newcommand{\ipa}[1]{{\ipafont #1 \regularfont}}

\usepackage{marginnote}

%% Observação: o pacote polyglossia pode apresentar erro ao ser utilizado com ifxetex + babel.
%% Se isso acontecer, atualize o pacote para a versão mais recente ou utilize somente uma das sequências (pdflatex ou xelatex), comentando ou apagando a outra.

\usepackage{microtype} 				% para melhorias de justificação
\usepackage[dvipsnames]{xcolor} 		% para cores
\usepackage{graphicx} 			% para imagens
\usepackage{booktabs,tabularx,rotating}	% para tabelas
\usepackage{mdframed} 				% para caixas de texto como na CIP do verso do título
\usepackage{multicol}				% tabelas com colunas mescladas
\usepackage{lettrine}				% letras capitulares
\usepackage{xspace} 				% para nao precisar de espaços com {} depois de comandos
									% como \LaTeX e abreviações criadas pelo usuário
\usepackage{lipsum} 				% para texto de preenchimento de exemplo
\usepackage{leading}				% espaçamento entrelinhas (leading)
\leading{13pt}

% ---
% Pacotes de citações
% ---
\usepackage[backend=biber,style=alphabetic,citestyle=alphabetic,backref=true]{biblatex}
\bibliography{refs}
\DefineBibliographyStrings{brazil}{
  backrefpage = {citado na página},
  backrefpages = {citado nas páginas},
}

% ---
% Informações do documento
% ---
\titulo{Computadores e suas Abstrações}
\autor{Gabriel V. N. de Queiroz}
\data{2019, v<VERSION>}
\preambulo{Breve sinopse do livro}
\local{São Paulo}
\instituicao{Publicações Acadêmicas Ltda.\\ \abnTeX\ v<VERSION>}

% alterando o aspecto da cor azul
\definecolor{blue}{RGB}{41,5,195}

% informações do PDF
\makeatletter
\hypersetup{
     	%pagebackref=true,
		pdftitle={\@title},
		pdfauthor={\@author},
    	pdfsubject={\imprimirpreambulo},
	    pdfcreator={LaTeX with abnTeX2},
		pdfkeywords={abnt}{latex}{abntex}{abntex2}{livro},
		colorlinks=false,       		% false: boxed links; true: colored links
    	linkcolor=blue,          	% color of internal links
    	citecolor=blue,        		% color of links to bibliography
    	filecolor=magenta,      		% color of file links
		urlcolor=blue,
		bookmarksdepth=4
}
\makeatother
\usepackage{xurl}
% ---


% ---
% Estilo de capítulos
%
% \chapterstyle{pedersen}
% \chapterstyle{lyhne}
%\chapterstyle{madsen}
\chapterstyle{veelo}
%
% Veja outros estilos em:
% https://www.ctan.org/tex-archive/info/MemoirChapStyles
% ---

% para cabeçalhos sem estar em maiúsculas
%\nouppercaseheads

% -----
% Declarações de cabecalhos
% -----
% Cabecalho padrao
\makepagestyle{abntbookheadings}
\makeevenhead{abntbookheadings}{\ABNTEXfontereduzida\thepage}{}{\ABNTEXfontereduzida\textit\leftmark}
\makeoddhead{abntbookheadings}{\ABNTEXfontereduzida\textit\rightmark}{}{\ABNTEXfontereduzida\thepage}
\makeheadrule{abntbookheadings}{\textwidth}{\normalrulethickness}

% Cabecalho do inicio do capitulo
\makepagestyle{abntbookchapfirst}
\makeoddhead{abntbookchapfirst}{}{}{}

% Configura layout para elementos textuais
\renewcommand{\textual}{%
  \pagestyle{abntbookheadings}%
  \aliaspagestyle{chapter}{abntbookchapfirst}% customizing chapter pagestyle
  \nouppercaseheads%
  \bookmarksetup{startatroot}%
}
% ---



% Margens do documento
%% (margens do abntex2 não combinam nem com A5 nem com estilos de capítulo da
% classe memoir.)
\setlrmarginsandblock{2.5cm}{3.5cm}{*}
\setulmarginsandblock{2.5cm}{3.5cm}{*}
\checkandfixthelayout
% ---

% Termos indexados
\usepackage{makeidx}
\makeindex
% \providecommand{\myglsindextext}[1]{\glsentrydesc{#1} (\glsentrytext{#1})}
% \providecommand{\myglsindex}[1]{\index{\myglsindextext{#1}}}
\renewcommand{\glsnamefont}[1]{{\normalfont \textsc{#1}}}
\newcommand{\glsstyle}[1]{{\textsc{#1}}}
\providecommand{\myglsindex}[1]{\index{#1@\glsstyle{#1}}}
\providecommand{\mygls}[1]{{\scshape \gls{#1}\myglsindex{#1}}}
\providecommand{\myglspl}[1]{{\scshape \glspl{#1}\myglsindex{#1}}}
\providecommand{\myGls}[1]{{\scshape \Gls{#1}\myglsindex{#1}}}
\providecommand{\myGlspl}[1]{{\scshape \Glspl{#1}\myglsindex{#1}}}

% Glossários
\makeglossaries
% \newglossaryentry{pai}{
%     name={pai},
%     plural={pai},
%     description={este é uma entrada pai, que possui outras
%     subentradas.} }
\newglossaryentry{bit}{
    name={bit},
    plural={bits},
    description={Dígito binário. Pode ser \texttt{0} (zero) ou \texttt{1} (um)} }
\newglossaryentry{byte}{
    name={byte},
    plural={bytes},
    description={(en: /\ipa{baɪt}/, pt-BR: /\ipa{baitʃi}/) Sequência de 8 bits} }
\newglossaryentry{palavra}{
    name={palavra},
    plural={palavras},
    description={(do inglês: \mygls{word}) Sequência de bits cujo tamanho é fixo e depende do processador em questão} }
\newglossaryentry{word}{
    name={word},
    plural={words},
    description={Veja \mygls{palavra}} }

\newglossaryentry{file}{
    name={file},
    plural={files},
    description={Veja \mygls{ficheiro}} }
\newglossaryentry{archive}{
    name={archive},
    plural={archives},
    description={Veja \mygls{arquivário}} }
\newglossaryentry{arquivo}{
    name={arquivo},
    plural={arquivos},
    description={Veja \mygls{ficheiro} e \mygls{arquivário}} }
\newglossaryentry{ficheiro}{
    name={ficheiro},
    plural={ficheiros},
    description={Unidade discreta de armazenamento geralmente organizada em pastas de forma hierárquica} }
\newglossaryentry{arquivário}{
    name={arquivário},
    plural={arquivários},
    description={(neologismo, do inglês: \mygls{archive}, mais frequentemente traduzido como \mygls{arquivo}) Coleção de registros, informações ou documentos. Ex: \emph{Arquivo Nacional}, \emph{Internet Archive} e \emph{National Archives and Records Administration}} }
\newglossaryentry{pasta}{
    name={pasta},
    plural={pastas},
    description={(do inglês: \mygls{folder}) Coleção de arquivos e outras pastas} }
\newglossaryentry{folder}{
    name={folder},
    plural={folders},
    description={Veja \mygls{pasta}} }
\newglossaryentry{diretório}{
    name={diretório},
    plural={diretórios},
    description={Veja \mygls{pasta}} }
\newglossaryentry{directory}{
    name={directory},
    plural={directorys},
    description={Veja \mygls{folder}} }

\newglossaryentry{float}{
    name={número de ponto flutuante},
    plural={números de ponto flutuante},
    description={Aproximação numérica análoga à notação científica} }
\newglossaryentry{num-flu}{
    name={número flutuante},
    plural={números flutuantes},
    description={Veja \mygls{float}} }
\newglossaryentry{lit-float}{
    name={float},
    plural={floats},
    description={Veja \mygls{float}} }
